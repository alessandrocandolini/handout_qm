\svnid{$Id: acknowledgements.tex 474 2010-07-23 01:07:53Z lsb $}%
\chapter{Acknowledgements}\newcommand{\tnk}{\textsc}
\lettrine{T}{his} thesis would never have been possible without the ideas, friendship, support and assistance of numerous people.  Foremost among them is of course \tnk{Steve Girvin}, who is infinitely more patient than the stereotypical infinitely-patient doctoral advisor.\footnote{Akin to \TeX\ \texttt{fill} versus \texttt{fil}.} He is a constant source of ideas and inspiration, and I feel enormously privileged to have him as my advisor.

I owe much to \tnk{Jens Koch}, who between numerous cups of coffee showed me how to write a scientific article. (I was slow to learn, so he taught me twice.)

My experimental colleague \tnk{Jerry Chow} went off on his own tangent to produce the beautiful data presented in \cref{ch:vrabi}. He then put up with my demands for ever more precise numbers to feed into my simulations, which he was able to provide without breaking off from his $l\times m\times n$ multitasking in $l$ windows on $m$ virtual desktops on $n$ monitors.

I have bounced many a crazy-sounding idea off \tnk{Andrew Houck} in order to gauge \emph{exactly how crazy} the idea may be. \tnk{Dave Schuster} has coefficient of restitution larger than unity: my crazy ideas bounce off him and come back at me \emph{much more crazy}. I can guarantee to overcome any mental blocks by talking to him for an hour.

\tnk{Eran Ginossar}, \tnk{Andreas Nunnenkamp} and \tnk{Lars Tornberg} each contributed aspects of the work presented in \cref{ch:ghz}. Our weekly (later, daily) group meetings and pairwise problem-solving sessions during that time are a fond memory of mine, despite my spending much of the time in a state of confusion.

\tnk{Jay Gambetta} advocated the quantum trajectories approach that proved very fruitful in \cref{ch:ghz}, and I learned much of what I know about circuit QED by talking to him and reading his papers.

I thank all of my friends, especially my friends on the 4th floor of Becton---the community that provided the experimental motivation for this thesis. It is a special environment where theorists and experimentalists can collaborate so closely.

Most importantly, I must thank my family, for their love and encouragement. It is impossible to record how grateful I am to my parents, nor can I imagine that I could have completed this thesis without the love of my wife and best friend~\tnk{June}. 