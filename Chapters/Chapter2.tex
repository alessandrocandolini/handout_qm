
%*******************************************************
% Chapter 2
%*******************************************************



\myChapter{Linear operators in Hilbert spaces}

Operator formulation of standard non-relativistic quantum mechanics heavily
relies on the
theory of linear operators in Hilber spaces.
In particular, the spectral theory of self-adjoint operators (bounded and
unbounded ones) is a key
ingredient in formulating the basic rules of quantum mechanics.
This chapter is aimed at providing the necessary mathematical background of functional
analysis employed by non-relativistic quantum mechanics. 
It is a chapter on mathematics, not on quantum physics.
The Reader interested in how functional analysis is applied to formulate quantum
mechanics should jump to the next chapters. 

\section{Banach and Hilber spaces}

Unless stated otherwise, let $\K $ denote equivalently the field of real numbers
$\R$ or the field of complex numbers $\C$.
(It is possible to develop the theory also for the skew-field of quaternions,
but this case will not be discussed here to avoid dealing with the
non-commutativity of quaternionic product.)

\begin{definition}[norm]
   Let $V$ be any linear space over $\K$.
   A ``norm'' on $V$ is any application  $V \to \R$  denoted by $\norm{\cdot}$ 
   %$\fullfunction{\norm{\cdot}}{V}{\R}$
   satifying the following properties:
   \begin{enumerate}[(a)]
      \item 
	 \label{item:norm1}
	 \begin{math}
	    \norm{\varphi} \geq 0  \condition{$\forall \varphi \in V$}
	 \end{math},
      \item 
	 \label{item:norm2}
	 $\norm{\varphi} = 0$ if and only if $\varphi = 0$,
      \item 
	 \label{item:norm3}
	 $\norm{\alpha \varphi} = \abs{\alpha} \norm{\varphi}$ for all
	 $\alpha \in \K$ and $\varphi \in V$,
      \item 
	 \label{item:norm4}
	 $\norm{\varphi + \psi} \leq \norm{
	    \varphi} + \norm{\psi}$, for all $\varphi$ and $\psi$ in $V$ (this
	 is called ``triangle inequality'')
   \end{enumerate}
\end{definition}

\begin{definition}[normed linear space]
   A ``normed linear space'' is a pair $(V, \norm{\cdot})$ where $V$ is a
   linear  space and $\norm{\cdot}$ is any norm on $V$.
\end{definition}

\begin{theorem}
   Let $(V, \norm{\cdot})$ be any normed linear space. 
   Let $\fullfunction{d}{V \times V}{\R}$ be the function defined by
   \begin{dmath}[label={distance-norm}]
      d( \varphi, \psi) = \norm{\varphi - \psi}
   \end{dmath}.
   Then,  $(V,d)$ is a metric space.
   The metric in \cref{eq:distance-norm} is called the ``metric induced by the
   norm'' on $V$.
\end{theorem}

\begin{proof}
   We need to prove that \cref{eq:distance-norm} defines a metric over $V$,
   \ie, we need to show that $d$  satisfies:
   \begin{enumerate}[(a)]
      \item 
	 \label{item:distance1}
	 \begin{math}
	    d(\varphi,\psi) \geq 0 \condition{$\forall (\varphi,\psi)\in V
	       \times V$}
	 \end{math},
      \item 
	 \label{item:distance2}
	 $d(\varphi,\psi) = 0$ if and only if $\varphi = \psi$,
      \item 
	 \label{item:distance3}
%	    $d(\varphi,\psi) = d(\psi,\varphi)$ for all 
	 \begin{math}
	    d(\varphi,\psi) = d(\psi,\varphi) \condition{$\forall (\varphi,\psi)\in V
	       \times V$}
	 \end{math},
      \item 
	 \label{item:distance4}
	 \begin{math}
	    d(\varphi,\psi) \leq  d(\psi,\eta) + d(\eta,
	    \psi)\condition{$\forall (\varphi,\psi,\eta)\in V
	       \times V\times V $}
	 \end{math} (the ``triangle inequality'').
   \end{enumerate}
   \Cref{item:distance1} follows from property~\ref{item:norm1} of the norm.
   \Cref{item:distance2} follows from property~\ref{item:norm2} of the norm,
   since
      $\norm{\varphi - \psi} = 0 $ if and only if $\varphi - \psi = 0$, \ie, if
      and only if $\varphi = \psi$.
      \Cref{item:distance3} follows from property~\ref{item:norm3} of the norm,
      since 
      \begin{dmath*}[compact]
	 d(\varphi, \psi) = 
      \norm{\varphi - \psi} = \norm { - ( \psi - \varphi)} = \abs{-1}
      \norm{\psi-\varphi} = \norm{\psi-\varphi} 
      = d(\psi,\varphi) 
   \end{dmath*}.
      \Cref{item:distance4}  follows from property~\cref{item:norm4} of the
      norm, since
      \begin{dmath*}[compact]
	 d(\varphi, \psi) = 
	 \norm{\varphi - \psi } = \norm{\varphi - \eta + \eta - \psi}
	 \leq \norm{\varphi - \eta} + \norm{\eta - \psi} 
	 = 
	 d(\varphi, \eta) + d(\eta, \psi) 
      \end{dmath*}.
      This completes the proof.
\end{proof}
