
%*******************************************************
% Chapter 2
%*******************************************************



\myChapter{Linear operators in Hilbert spaces}

Operator formulation of non-relativistic quantum mechanics heavily relies on th
theory of linear operators in Hilber spaces.
In particular, the spectral theory of self-adjoint operators is a key
ingredient in formulating the basic rules of quantum mechanics.
This chapter is aimed at providing the necessary mathematical background of functional
analysis employed by quantum mechanics. 
It is a chapter on mathematics, not on quantum physics.
The Reader interested in how functional analysis is used to formulate quantum
mechanics should jump to the next chapters. 

\section{Banach and Hilber spaces}

Unless stated otherwise, let $\K $ denote equivalently the field of real numbers
$\R$ or the of complex numbers $\C$.

\begin{definition}[norm]
   Let $V$ be any linear space over $\K$.
   A norm on $V$ is any application $\fullfunction{\norm{\cdot}}{V}{\R}$
   satifying
   \begin{enumerate}[(a)]
      \item $\norm{\varphi} \geq 0 $ for any $\varphi \in V$, 
      \item $\norm{\varphi} = 0$ if and only if $\varphi = 0$,
      \item $\norm{\alpha \varphi} = \abs{\alpha} \norm{\varphi}$ for all
	 $\alpha \in \K$ and $\varphi \in V$,
      \item $\norm{\varphi + \psi} \leq \norm{
	    \varphi} + \norm{\psi}$, for all $\varphi$ and $\psi$ in $V$ (this
	 is called ``triangle inequality'')
   \end{enumerate}
\end{definition}
