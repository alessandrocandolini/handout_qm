%*******************************************************
% Prefazione 
%*******************************************************


\myChapter*{Aphorisms}
\markboth{\spacedlowsmallcaps{Aphorisms}}{\spacedlowsmallcaps{Aphorisms}} 
%\mtcaddchapter[\numberline{}\tocEntry{Aphorisms}]

\addcontentsline{toc}{chapter}{\numberline{}\tocEntry{Aphorisms}}


The \emph{lietemotiv} is \emph{renormalization}. Let us summarize the key
ideas. The section is mainly inspired by the introductory chapter of
Zinn-Justin.

Relativistic quantum field theory (QFT) was originally developed to match
quantum mechanics with the principles of special relativity. 
Now, it offers the most comprehensive theoretical framework to discuss
elementary excitations above the ground state of physical systems with an
infinite number of degrees of freedom.

In its various incarnations, QFT is the theoretical framework employed to
describe all foundamental interactions but
gravitations at microscopic scale --- in the Standard Model of Particle Physics,
the strong interactions  are described by  Quantum Chromodynamics (familiarly
called QCD), which is 
an unbroken SU(3)$_{c}$ locally-symmetric Yang-Mills gauge QFT, while the electroweak
interactions are described by  a spontaneusly broken SU(2)$_{w}
\times$U(1)$_{w}$ Yang-Mills gauge QFT ---  and
let to a deep understanding of the singular properties of a wide class of phase
transitions at the critical point; furthermore, the statistical properties of
some geometrical models (\eg, self-avoiding random walks, which are of
practical interest) can be deal with by using the tools of QFT.

However, QFT in its most direct formulations, comes  with a \emph{severe conceptual
   drawback}, namely the appearance of infinities in the
calculation  of physical quantities. 

To avoid the occurrence of infinities, an empirical, somewhat \emph{ad hoc} (but
systematic) procedure,
which goes under the name of renormalization, was eventually developed which allows extracting from
(meaningless) divergent mathematical expressions
(meaningfull) finite numerical predictions to be compared with experiments.

The renormalization recipe works in this spirit: you have to carefully
distinguish between \emph{bare} quantities (\eg, mass, electric charge) and
\emph{effective}
quantities. The latter ones are those which actually have to be related to
experiments, while the former ones are additional auxiliary parameters of the
physical model, which are necessary in order the renormalization recipe to work
but whose physical meaning remains somewhat obscured.
At this level, it might be said that renornalization is a way of organizing the
calculations in order to get finite results from expressions which naively
speaking would otherwise led to infinities.

From a pragmatic point of view, 
renormalization works: It 
has allowed and still allows calculations of increasing precision.
In fact, the renormalization procedure would hardly have been convincing if the
predictions were not confirmed with increasing precision by experiments.

Only later, did the procedure find a satisfactory physical interpretation which
clarify the deep origin of renormalization and enlight the role of
renormalizable QFTs.
   The problem of infinities is related to an unexpected phenomenon:
   \emph{Renormalization is related to the non-decoupling of very different
      lenght scales}
Today, Those who are familiar with Kenneth Wilson's ideas and the renormalization
group, will immediately say that actually there is no divergence.
More or less, the story goes like this: 
Every QFT requires an ultraviole completion (thus being an effective theory).
The only difference between renormalizable and non-renormalizable QFTs lies in
the fact that the former are \emph{insensitive} to the ultraviolet data (which
can be absorbed in a few low-energy parameters) while the latter depend on the
details of the ultraviole completion.



 
