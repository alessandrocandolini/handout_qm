
% *****************************************************************************
% 	General commands
% *****************************************************************************

% un ambiente ad hoc per le citazioni
\newenvironment{citazione}%
  {\begin{quotation}\small\ignorespaces}%
  {\end{quotation}}

% [...] ;-)
\newcommand{\omissis}{[\dots\negthinspace]}

% Indica a LaTeX la cartella dove sono riposti i file delle immagini
\graphicspath{{./},{./Asymptote/}, {./Images/}, {./Metapost/}}

% Eccezioni all'algoritmo di sillabazione
\hyphenation{Fortran ma-cro-istru-zio-ne nitro-idrossil-amminico}

% Un ambiente ad hoc per gli approfondimenti 
\newenvironment{approfondimento}%
  {\begin{quotation}\small\ignorespaces}%
  {\end{quotation}}

% i.e.
\newcommand{\ie}{i.\,e.}
\newcommand{\Ie}{I.\,e.}
\newcommand{\eg}{e.\,g.}
\newcommand{\Eg}{E.\,g.} 

% th
\newcommand{\ordth}{\textsuperscript{th}}

% Cpp 
\def\cpp{\mbox{C++}}

% Additional minitoc stuff
\newif\ifminitoc
\makeatletter
\@ifpackageloaded{minitoc}{\minitoctrue}{\minitocfalse}
\ifminitoc
  \let\myoldminitoc\minitoc
  \def\minitoc{
     \myoldminitoc\mtcskip
     \@afterheading
  }
\fi
\makeatother


% *****************************************************************************
% 	Additional hyperref stuff
% *****************************************************************************
\hypersetup{%
    colorlinks=true, linktocpage=true, pdfstartpage=1, pdfstartview=FitV,%
    breaklinks=true, pdfpagemode=UseNone, pageanchor=true, pdfpagemode=UseOutlines,%
    plainpages=false, bookmarksnumbered, bookmarksopen=true, bookmarksopenlevel=1,%
    hypertexnames=true, pdfhighlight=/O,%
    urlcolor=webbrown, linkcolor=RoyalBlue, citecolor=RoyalBlue, pagecolor=RoyalBlue,%
   hyperfootnotes=true,%
% uncomment the following line if you want to have black links (e.g., for printing)
% urlcolor=Black, linkcolor=Black, citecolor=Black, pagecolor=Black,%
    pdftitle={\myTitle},%
    pdfauthor={\textcopyright\ \myName},%
    pdfsubject={},%
    pdfkeywords={},%
    pdfcreator={pdfLaTeX},%
    pdfproducer={LaTeX con hyperref e ClassicThesis}%
}


\hypersetup{citecolor=webblue}

\newcommand{\mail}[1]{\href{mailto:#1}{\texttt{#1}}}

% *****************************************************************************
% 	Mathematica Packages
% *****************************************************************************

% A.M.S. standard packages.
\usepackage{amssymb}

% Numerical sets (amssymb package required).
\newcommand{\numberset}{\mathbb} 
\newcommand{\N}{\numberset{N}} 
\newcommand{\Z}{\numberset{Z}} 
\newcommand{\Q}{\numberset{Q}} 
\newcommand{\R}{\numberset{R}} 
\newcommand{\C}{\numberset{C}} 

% Dirac notation (braket package required).
\usepackage{braket} 
\usepackage{functan} 
\newcommand{\modul}[1]{\mathinner{\vert#1\vert}} 
\newcommand{\Modul}[1]{\left\vert#1\right\vert} 
%\newcommand{\norm}[1]{\mathinner{\Vert#1\Vert}}
%\newcommand{\Norm}[1]{\left\Vert#1\right\Vert}
\newcommand{\conj}[1]{#1^{*}} 
\newcommand{\adj}[1]{#1^{\dagger}}

% Un ambiente per i sistemi.
\newenvironment{sistema}%
  {\left\lbrace\begin{array}{@{}l@{}}}%
  {\end{array}\right.}

% Cool package.
\usepackage{cool}
\makeatletter
  \Style{ArcTrig=arc}
  \Style{DSymb={{\operator@font d}}}
  \Style{DDisplayFunc=inset,DShorten=true}
  \Style{IntegrateDifferentialDSymb={{\operator@font d}}}
\makeatother

% 12many package. 
\usepackage{12many}
\newOTMstyle[var={k},dots={, \ldots, }]{mydots}{%
   \getOTMparameter{var}=#1\getOTMparameter{dots}#2%
}
\setOTMstyle{mydots}

% Intervals
\makeatletter
\DeclareRobustCommand{\genericinterval}[2]{%
  \@ifstar{\genericinterval@star{#1}{#2}}{\genericinterval@nostar{#1}{#2}}}
\newcommand{\genericinterval@star}[4]{\mathopen{}\mathclose{\left#1#3,#4\right#2}}
\newcommand{\genericinterval@nostar}[4]{\mathopen{#1}#3,#4\mathclose{#2}}
\newcommand{\intervalcc}{\genericinterval[]}
\newcommand{\intervaloo}{\genericinterval][}
\newcommand{\intervaloc}{\genericinterval]]}
\newcommand{\intervalco}{\genericinterval[[}
\newcommand{\cinterval}{\genericinterval[]}
\newcommand{\ointerval}{\genericinterval][}
\newcommand{\commutator}{\genericinterval[]}
\makeatother

% *****************************************************************************
% 	Cleveref support for breqn
% *****************************************************************************
\usepackage{cleveref}

% *****************************************************************************
% 	Theorem and Definition Environments.
% *****************************************************************************

\usepackage{amsthm}                       % A.M.S. package for theorems.

\makeatletter
  \newtheoremstyle{classicdef}%           % Stile tipografico dei teoremi
  {11pt}%                                 % Spazio che precede l'enunciato
  {11pt}%                                 % Spazio che segue l'enunciato
  {}%                                     % Stile del font dell'enunciato
  {}%                                     % Rientro (se vuoto, nessun rientro;
  %                                       % \parindent = rientro dei capoversi)
  {\scshape}%                             % Font dell'intestazione
  {:}%                                    % Punteggiatura dopo l'intestazione
  {.5em}%                                 % Spazio che segue l'intestazione:
  %                                       % " " = normale spazio inter-parola;
  %                                       % \newline = a capo
  {}%                                     % Specifica intestazione enunciato
\makeatother

\theoremstyle{classicdef}
\newtheorem{theorem}{Theorem}[chapter]
\newtheorem{lemma}{Lemma}[chapter]
\newtheorem{definition}{Definition}[chapter]
\newtheorem*{homework}{Homework}
\theoremstyle{remark}
\newtheorem*{remark}{Remark}
\renewcommand{\qedsymbol}{\rule{.5em}{.5em}}

% *****************************************************************************
% 	Exercise environment 
% *****************************************************************************

\usepackage[framemethod=TikZ]{mdframed}

\tikzstyle{titregris} =
          [draw=gray, thick, fill=white, shading = exercisetitle, %
           text=gray, rectangle, rounded corners,
           right,minimum height=.7cm]

\usetikzlibrary{shadows}

\pgfdeclarehorizontalshading{exercisebackground}{100bp}
{color(0bp)=(brown!40);
color(100bp)=(black!5)}

\pgfdeclarehorizontalshading{exercisetitle}{100bp}
{color(0bp)=(red!40);
color(100bp)=(black!5)}

\makeatletter

\def\mdf@@exercisepoints{}
\define@key{mdf}{exercisepoints}{%
    \def\mdf@@exercisepoints{#1}
}

\mdf@do@stringoption{problemtitle=={Problema}}

\mdfdefinestyle{exercisestyle}{%
  singleextra={
      \node[titregris,xshift=1cm] at (P-|O) %
           {~\mdf@frametitlefont{\mdf@problemtitle}~};
      \ifdefempty{\mdf@@exercisepoints}%
      {}%
      {\node[titregris,left,xshift=-1cm] at (P)%
        {~\mdf@frametitlefont{\mdf@@exercisepoints points}~};}%
  },
  firstextra={
      \node[titregris,xshift=1cm] at (P-|O) %
           {~\mdf@frametitlefont{\mdf@problemtitle}~};
      \ifdefempty{\mdf@@exercisepoints}%
      {}%
      {\node[titregris,left,xshift=-1cm] at (P)%
        {~\mdf@frametitlefont{\mdf@@exercisepoints points}~};}%
	   \node[right,xshift=6cm, yshift=0.2cm] at (O)%
      {{\footnotesize (continua nella prossima pagina\ldots)}~};
  },
  secondextra={
      \node[titregris,xshift=1cm] at (P-|O) %
           {~\mdf@frametitlefont{\mdf@problemtitle}~};
      \node[titregris,left,xshift=-1cm] at (P)%
      {~\mdf@frametitlefont{\small (continuation)}~};%
  },
  middleextra={
      \node[titregris,xshift=1cm] at (P-|O) %
           {~\mdf@frametitlefont{\mdf@problemtitle}~};
      \node[titregris,left,xshift=-1cm] at (P)%
      {~\mdf@frametitlefont{\small (continuation)}~};%
  },
  footnoteinside=true,
  outerlinewidth=1pt,
  innerlinewidth=1pt,
  middlelinewidth=1pt,
  shadow=true,
  roundcorner=3pt,
  linecolor=gray,
  tikzsetting={shading = exercisebackground},
  innertopmargin=1.2\baselineskip,
  skipabove={\dimexpr0.5\baselineskip+\topskip\relax},
  splittopskip=0.8cm,
  splitbottomskip=0.8cm,
  needspace=3\baselineskip,
  frametitlefont=\scshape,%
  %settings={\global\stepcounter{problem}},
}

\makeatother

\providecommand{\problemname}{Problem}
\newcounter{problem}[chapter]
\setcounter{problem}{0}

\definecolor{hilite}{rgb}{0.2,0.4,0.7}
\colorlet{problemcolor}{hilite}

\makeatletter
\renewcommand{\theproblem}{%
    \ifnum \c@chapter>\z@ \thechapter.\fi \@arabic\c@problem
}

\def\Exercise{%
  \@ifnextchar [%
      {\@myprobt}%
      {\@myprob}%
}
  

\def\@myprobt[#1]{%
\refstepcounter{problem}
\begin{mdframed}[%
     skipabove=12pt,%
     skipbelow=12pt,%
     style=exercisestyle,%
     frametitlefont=\scshape,%
     problemtitle=\problemname~\theproblem{}~(#1)
%     frametitle=Problem~\theproblem{}
]
\begingroup
}

\def\@myprob{%
\refstepcounter{problem}
\begin{mdframed}[%
     skipabove=10pt,%
     skipbelow=10pt,%
     style=exercisestyle,%
     frametitlefont=\scshape,%
     problemtitle=\problemname~\theproblem{}
]
\begingroup
}

\makeatother

\def\endExercise{\endgroup\end{mdframed}}

% *****************************************************************************
% 	Solution environment 
% *****************************************************************************
\newcommand*{\myfancysym}{\color{hilite}\star}

\newcommand*{\fancybreak}{%
   \par
   \nopagebreak\medskip\nopagebreak
   \noindent\null\hfill$\myfancysym\quad\myfancysym\quad\myfancysym\quad$\hfill\null\par
   \nopagebreak\medskip\pagebreak[0]%
}

\newenvironment{Solution}%{%
{%
%%\normalfont
%%\par
%%\noindent
%%{\scshape Soluzione.}
%%}
   \trivlist\item[\hskip\labelsep{\scshape Soluzione.}]
}%
{%
   \fancybreak%
   \endtrivlist%
}

% *****************************************************************************
% 	Digression environment
% *****************************************************************************
\usetikzlibrary{calc,arrows}
\tikzset{
    excursus arrow/.style={%
       line width=2pt,
       draw=gray!40,
       rounded corners=2ex,
    },
    excursus head/.style={
       fill=white,
       font=\bfseries\sffamily,
       text=gray!80,
       anchor=base west,
    },
 }
 
 \makeatletter
 \mdf@do@stringoption{digressiontitle=={Digression}}
 \mdfdefinestyle{digressionarrows}{%
  singleextra={%
       \path let \p1=(P), \p2=(O) in (\x2,\y1) coordinate (Q);
       \path let \p1=(Q), \p2=(O) in (\x1,{(\y1-\y2)/2}) coordinate (M);
       %\path [excursus arrow, round cap-to]
       \path [excursus arrow]
          ($(O)+(5em,0ex)$) -| (M) |- %
          %($(Q)+(12em,0ex)$) .. controls +(0:16em) and +(185:6em) .. %
          %++(23em,2ex);
 	 ($(Q)+(12em,0ex)$) --  (P);
          %+(13em,0ex);
       \node [excursus head] at ($(Q)+(2.5em,-0.75pt)$) {%
 	 \mdf@frametitlefont{\mdf@digressiontitle}};},
  firstextra={%
       \path let \p1=(P), \p2=(O) in (\x2,\y1) coordinate (Q);
       \path let \p1=(Q), \p2=(O) in (\x1,{(\y1-\y2)/2}) coordinate (M);
       \path [excursus arrow]
          %(O) |- %
          ($(O)+(5em,0ex)$) -| (M) |- %
          %($(Q)+(12em,0ex)$) .. controls +(0:16em) and +(185:6em) .. %
          %++(23em,2ex);
 	 ($(Q)+(12em,0ex)$) --  (P);
       \node [excursus head] at ($(Q)+(2.5em,-2pt)$) {%
 	 \mdf@frametitlefont{\mdf@digressiontitle}};},
       %\node [excursus head] at ($(Q)+(2.5em,-2pt)$) {Digression};},
  secondextra={%
       %\path let \p1=(P), \p2=(O) in (\x2,\y1) coordinate (Q);
       %\path [excursus arrow,round cap-]
       %   ($(O)+(5em,0ex)$) -| (Q);};
      \path let \p1=(P), \p2=(O) in (\x2,\y1) coordinate (Q);
      \path let \p1=(Q), \p2=(O) in (\x1,{(\y1-\y2)/2}) coordinate (M);
      \path [excursus arrow]
         %%(O) |- %
         ($(O)+(5em,0ex)$) -| (M) |- %
         %($(Q)+(12em,0ex)$) .. controls +(0:16em) and +(185:6em) .. %
         %++(23em,2ex);
  ($(Q)+(12em,0ex)$) --  (P);
  \node [excursus head] at ($(Q)+(2.5em,-2pt)$) {\mdf@frametitlefont{\mdf@digressiontitle{}
	(continuation)}};},
 middleextra={%
      \path let \p1=(P), \p2=(O) in (\x2,\y1) coordinate (Q);
      \path [excursus arrow]
          (O) -- (Q);},
%middlelinewidth=2.5em,middlelinecolor=white,
   hidealllines=true,
   %topline=true,
   innertopmargin=2.5ex,
   innerbottommargin=2.5ex,
   innerrightmargin=0ex,
   innerleftmargin=2ex,
   %skipabove=0.87\baselineskip,
   % skipbelow=0.62\baselineskip,
  %needspace=2\baselineskip,
 % splittopskip=2.5ex,
  % splitbottomskip=2.5ex,
}
 
\makeatother

 
\newenvironment{digression}[1][Digressione]%
{
   \begin{mdframed}[%
   style=digressionarrows, 
   frametitlefont=\scshape,
   digressiontitle=#1]
   \begingroup
}
{
\endgroup
\end{mdframed}
}


% *****************************************************************************
% 	Vectors 
% *****************************************************************************
\usepackage[veceuler]{utvec}
\MakeRobust\vec


% *****************************************************************************
% 	Biblatex 
% *****************************************************************************

\renewcommand{\nameyeardelim}{, }

\defbibheading{bibliography}{%
\cleardoublepage
\manualmark
\phantomsection
%\ifminitoc
\mtcaddchapter[\numberline{}\tocEntry{\bibname}]
%\else
%\addtocontents{toc}{\protect\vspace{\beforebibskip}}
%\addcontentsline{toc}{chapter}{\numberline{}\tocEntry{\bibname}}
%\fi
\myChapter*{\bibname\markboth{\spacedlowsmallcaps{\bibname}}
{\spacedlowsmallcaps{\bibname}}}}     

% *****************************************************************************
% 	Caption
% *****************************************************************************
\usepackage{caption}                      % Fancy captions and more.
\DeclareCaptionFont{captioncolor}{\color{hilite}}
\captionsetup{format=plain,labelsep=period,font=small, labelfont={sc,
      captioncolor}}
\captionsetup[table]{skip=\medskipamount} 


% *****************************************************************************
% 	Makeidx, Multicol
% *****************************************************************************
\usepackage{varindex}
\usepackage{toolbox}
\let\orgtheindex\theindex
\let\orgendtheindex\endtheindex
\def\theindex{%
	\def\twocolumn{\begin{multicols}{2}}%
	\def\onecolumn{}%
	\clearpage
	\orgtheindex
}
\def\endtheindex{%
	\end{multicols}%
	\orgendtheindex
}

\makeindex

% *****************************************************************************
% 	Breqn
% *****************************************************************************
\usepackage{mathtools}                    % Add support for cramped,
					  % mathlap,etc.
%\usepackage{siunitx}                      % Add support to print SI units
\usepackage[euler]{flexisym}              % Add support to Euler font
\usepackage{breqn}                        % Breqn

\makeatletter
   \def\eqnumsize{\normalfont \Tf@font}   % Add support to Minion Pro
\makeatother

\setkeys{breqn}{labelprefix={eq:}}

\catcode`\_=8 % for safety


\newcommand{\breakingcomma}{%
  \begingroup
  \lccode`~=`,
  \lowercase{\endgroup\expandafter\def\expandafter~\expandafter{~\penalty0 }}}

% *****************************************************************************
% 	Cleveref support for breqn
% *****************************************************************************

\makeatletter
\renewcommand\set@label[2]{\protected@edef\@currentlabel{#2}
\cref@constructprefix{equation}{\cref@result}%
\protected@xdef\cref@currentlabel{%
[equation][\arabic{equation}][\cref@result]\eq@number}
  %%% Work in progress... Support for hyperref 
  \hyper@makecurrent{equation}
  \Hy@raisedlink{\hyper@anchorstart{\@currentHref}}
  \Hy@raisedlink{\hyper@anchorend}
}
\makeatother

      

% *****************************************************************************
% 	Migliorare il kerning dell'apostrofo coi font MinionPro
% *****************************************************************************
\makeatletter 
\catcode`\'=12 
\def\qu@te{'} 
\catcode`'=\active 
\begingroup 
\obeylines\obeyspaces% 
\gdef\@resetactivechars{% 
\def^^M{\@activechar@info{EOL}\space}% 
\def {\@activechar@info{space}\space}% 
}% 
\endgroup 
\providecommand{\texorpdfstring}{\@firstoftwo} 
\protected\def'{\texorpdfstring{\active@quote}{\qu@te}} 
\def\active@quote{\relax 
  \ifmmode 
    \expandafter^\expandafter\bgroup\expandafter\prim@s 
  \else 
    \expandafter\futurelet\expandafter\@let@token\expandafter\qu@t@ 
  \fi} 
\def\qu@t@{% 
  \ifx'\@let@token 
    \qu@te\qu@te\expandafter\@gobble 
  \else 
    {}\qu@te{}\penalty\@M\hskip\expandafter\z@skip 
  \fi} 
\scantokens\expandafter{% 
  \expandafter\def\expandafter\pr@m@s\expandafter{\pr@m@s}} 
\makeatother

% *****************************************************************************
%	 Margins
% *****************************************************************************
% Suggestions from ClassicThesis
% Palatino 	10pt: 288--312pt | 609--657pt
% Palatino 	11pt: 312--336pt | 657--705pt
% Palatino 	12pt: 360--384pt | 768pt

\areaset[current]{336pt}{750pt}
\setlength{\marginparwidth}{7em}
\setlength{\marginparsep}{2em}%

% *****************************************************************************
% Change color of section titles 
% *****************************************************************************

 \colorlet{seccolor}{brown}
 %\colorlet{seccolor}{OliveGreen}

%\definecolor{hilite}{rgb}{0.2,0.4,0.7}
% \colorlet{seccolor}{hilite}
% sections \FloatBarrier
    \titleformat{\section}
    {\relax}{\color{seccolor}\textsc{\MakeTextLowercase{\thesection}}}{1em}{\color{seccolor}\spacedlowsmallcaps}
    % subsections
    \titleformat{\subsection}
    {\relax}{\color{seccolor}\textsc{\MakeTextLowercase{\thesubsection}}}{1em}{\color{seccolor}\normalsize\itshape}
    % subsubsections
    \titleformat{\subsubsection}
    {\relax}{\color{seccolor}\textsc{\MakeTextLowercase{\thesubsubsection}}}{1em}{\color{seccolor}\normalsize\itshape}        
    % descriptionlabels
    \renewcommand{\descriptionlabel}[1]{\color{seccolor}\hspace*{\labelsep}\spacedlowsmallcaps{#1}}   % spacedlowsmallcaps textit textsc                  


% *****************************************************************************
% 	Additional setup and packages
% *****************************************************************************

\hypersetup{citecolor=webbrown}
\hypersetup{pdfstartpage=1}

\usepackage{nicefrac}
\usepackage{calligra}
\usepackage{lettrine}
\usepackage{paralist}
\usepackage{cancel}

\usepackage{asymptote}       % Asymptote Graphical Vector Language.
\begin{asydef}
defaultpen(fontsize(11pt));
texpreamble("\usepackage[opticals,onlytext]{MinionPro}");
texpreamble("\usepackage[small]{eulervm}");
texpreamble("\usepackage{nicefrac}");
texpreamble("\usepackage{cool}");
\end{asydef}

\newcommand*\openquote{\makebox(25,-12){\color{lightgray}\scalebox{5}{``}}}
\newcommand*\closequote{\makebox(25,-22){\color{lightgray}\scalebox{5}{''}}}

\Crefname{problem}{L'esercizio}{Gli esercizi}%
\crefname{problem}{l'esercizio}{gli esercizi}%

% *****************************************************************************
% 	Footmisc
% *****************************************************************************

\usepackage[perpage, symbol*, stable, multiple]{footmisc}	
\setlength{\footnotemargin}{.6em}%

\makeatletter
\newcounter{Hfootnote}%
  \let\H@@footnotetext\@footnotetext
  \let\H@@footnotemark\@footnotemark
  \def\@xfootnotenext[#1]{%
    \begingroup
      \csname c@\@mpfn\endcsname #1\relax
      \unrestored@protected@xdef\@thefnmark{\thempfn}%
    \endgroup
    \ifx\@footnotetext\@mpfootnotetext
      \expandafter\H@@mpfootnotetext
    \else
      \expandafter\H@@footnotetext
    \fi
  }%
  \def\@xfootnotemark[#1]{%
    \begingroup
      \c@footnote #1\relax
      \unrestored@protected@xdef\@thefnmark{\thefootnote}%
    \endgroup
    \H@@footnotemark
  }%
  \let\H@@mpfootnotetext\@mpfootnotetext
  \long\def\@mpfootnotetext#1{%
    \H@@mpfootnotetext{%
      \ifHy@nesting
        \expandafter\hyper@@anchor\expandafter{%
          \Hy@footnote@currentHref
         }{#1}%
      \else
        \Hy@raisedlink{%
          \expandafter\hyper@@anchor\expandafter{%
            \Hy@footnote@currentHref
          }{\relax}%
        }#1%
      \fi
    }%
  }%
  \long\def\@footnotetext#1{%
    \H@@footnotetext{%
      \ifHy@nesting
        \expandafter\hyper@@anchor\expandafter{%
          \Hy@footnote@currentHref
        }{#1}%
      \else
        \Hy@raisedlink{%
          \expandafter\hyper@@anchor\expandafter{%
            \Hy@footnote@currentHref
          }{\relax}%
        }%
        \let\@currentHref\Hy@footnote@currentHref
        \let\@currentlabelname\@empty
        #1%
      \fi
    }%
  }%
  \def\@footnotemark{%
    \leavevmode
    \ifhmode\edef\@x@sf{\the\spacefactor}\nobreak\fi
    \stepcounter{Hfootnote}%
    \global\let\Hy@saved@currentHref\@currentHref
    \hyper@makecurrent{Hfootnote}%
    \global\let\Hy@footnote@currentHref\@currentHref
    \global\let\@currentHref\Hy@saved@currentHref
    \hyper@linkstart{link}{\Hy@footnote@currentHref}%
    \@makefnmark
    \hyper@linkend
    \ifhmode\spacefactor\@x@sf\fi
    \relax
  }%


\long\def\@footnotetext#1{%
      \H@@footnotetext{%
        \ifHy@nesting
         \hyper@@anchor{\@currentHref}{#1}%
       \else
         \Hy@raisedlink{\hyper@@anchor{\@currentHref}{\relax}}#1%
       \fi
     }}


  \def\@footnotemark{%
     \leavevmode
     \ifhmode\edef\@x@sf{\the\spacefactor}\nobreak\fi
     \H@refstepcounter{Hfootnote}%
     \hyper@makecurrent{Hfootnote}%
     \hyper@linkstart{link}{\@currentHref}%
     \@makefnmark
     \hyper@linkend
     \ifhmode\spacefactor\@x@sf\fi
     \relax
   }%

 \ifFN@multiplefootnote%
     \renewcommand*\@footnotemark{%
      \leavevmode
      \ifhmode
        \edef\@x@sf{\the\spacefactor}%
        \FN@mf@check
        \nobreak
      \fi
      \H@refstepcounter{Hfootnote}%
      \hyper@makecurrent{Hfootnote}%
      \hyper@linkstart{link}{\@currentHref}%
      \@makefnmark
      \hyper@linkend
      \FN@mf@prepare
      \ifhmode\spacefactor\@x@sf\fi
      \relax%
    }%
 \fi

\makeatother

% *****************************************************************************
% 	Altro
% *****************************************************************************	
\def\eqspace{\,}
\def\erf{\mathrm{erf}}
\makeatletter
\newcommand{\Res}[0]{\mathop{\operator@font Res}\nolimits}
\newcommand{\torder}[0]{%\mathop{\operator@font T}\nolimits}
\mathop{\kern\z@\operator@font T}%
\csname nolimits@\endcsname}
\makeatother
\def\mat#1{\mathrm{#1}}
\def\Ltwo{L}
\usepackage{siunitx}
\newcommand{\tend}[0]{t^{\prime\prime}}
\newcommand{\tstart}[0]{t^{\prime}}
\newcommand{\qstart}[0]{q^{\prime}}
\newcommand{\qend}[0]{q^{\prime\prime}}
\newcommand{\pos}[0]{x}
\newcommand{\n}[0]{N}
\newcommand{\openone}{\IdentityMatrix}
\newcommand{\propagator}[4]{K\left( #1, #2; #3, #4 \right)}

